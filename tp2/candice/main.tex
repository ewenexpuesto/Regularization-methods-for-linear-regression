\documentclass[12pt,a4paper]{report}

\title{Rapport TP2}
\author{Candice AUBERTIN - Ewen EXPUESTO}
\date{20 Octobre 2025}

\usepackage[a4paper, margin=1.5cm]{geometry}
\usepackage{graphicx}
\usepackage{listings}
\usepackage{amsmath}
\usepackage{color}
\usepackage{xcolor}
\usepackage{float}
\usepackage[utf8]{inputenc} % handles UTF-8 characters
\usepackage[T1]{fontenc} % better font encoding for accented characters

\setcounter{section}{0}
\renewcommand{\thesection}{\arabic{section}}

\definecolor{backgroundColour}{rgb}{0.1, 0.1, 0.1}
\definecolor{bordeaux}{rgb}{0.5, 0, 0}
\definecolor{blueVar}{rgb}{0.0, 0.0, 1.0}
\definecolor{greenKeyword}{rgb}{0.0, 0.5, 0.0}

\usepackage{minted}
\setminted{fontsize=\small, breaklines, frame=single, bgcolor=gray!5}

\begin{document}
\maketitle

\tableofcontents

\newpage
\renewcommand{\contentsname}{Table des matières}

\newpage

\section{Electricity Data Set}


Le but est d'expliquer la production totale quotidienne d'électricité au Mexique à l'aide des variables contenues dans \texttt{Mexico\_data.csv}.

\subsection{Description des variables}
\begin{itemize}
    \item X0 : jour de l'année
    \item RH : humidité relative (\%)
    \item SSRD : rayonnement solaire incident à la surface (J.m$^{-2}$)
    \item STRD : rayonnement thermique incident à la surface (J.m$^{-2}$)
    \item T2M : température moyenne quotidienne à 2m (°C)
    \item T2Mmax : température maximale quotidienne à 2m (°C)
    \item T2Mmin : température minimale quotidienne à 2m (°C)
    \item Covid : indice de rigueur COVID-19
    \item Holidays : jours fériés, 1 = jour férié, 0 sinon
    \item DOW : jour de la semaine, 0 = lundi, 1 = mardi, ...
    \item TOY : jour de l'année (1 à 366)
    \item Total : production quotidienne d'électricité (GWh)
\end{itemize}

\subsection{Régression Ridge}

L'objectif est d'analyser les relations entre la variable cible day et un ensemble de prédicteurs présents dans le tableau de données \texttt{Mexico\_data.csv}. Nous allons donc construire un modèle de régression capable de prédire notre variable cible \texttt{Total} tout en limitant les problèmes dus à la présence d'un grand nombre de variables. \\
Cette méthode nous permet de pénaliser les grandes valeurs des coefficients, afin de stabiliser l'estimation des paramètres et d'améliorer la généralisation du modèle.

\[
\Phi(\beta) = \|Y - X\beta\|_2^2 + k \|\beta\|_2^2 
\qquad \text{($k \in \mathbb{R}_+^*$)}
\]
c'est à dire
\[
\Phi(\beta) = (Y - X\beta)^T (Y - X\beta) + k \sum_{j=1}^p \beta_j^2
\qquad \text{($k \in \mathbb{R}_+^*$)}
\]
\begin{minted}{r}
library(MASS)
library(glmnet)
tab <- read.csv(file = "Mexico_data.csv", header = TRUE, sep = ",")
Y <- as.numeric(tab$Total)
X <- as.matrix(tab[,-1])
Xs <- scale(X)
tabr <- data.frame(Y, Xs)
rm<-lm.ridge(Y~.,data=tabr,lambda=seq(0,20,by=0.1))
\end{minted}

Cependant, le logiciel crash à l'exécution de cette dernière commande. C'est pourquoi nous allons utiliser la librairie \texttt{glmnet}:

\begin{minted}{r}
Y_matrix <- matrix(Y, length(Y), 1)
m_glm<-cv.glmnet(Xs, Y_matrix, alpha = 0)
print(m_glm$lambda.min)
\end{minted}

\texttt{[1] 9.505062} \\
Avec \texttt{cv.glmnet} qui permet de minimiser la fonction $\Phi$\\
On ajuste donc le modèle avec le $\lambda$ obtenu:

\begin{minted}{r}
m_adjusted<- glmnet(Xs, Y_matrix, alpha = 0, lambda = m_glm$lambda.min)
\end{minted}

On \texttt{plot} la validation croisée pour la régression Ridge:

\begin{minted}{r}
mg<- cv.glmnet(Xs, Y, alpha = 0)
plot(mg)
abline(v = log(mg$lambda.min), col = "blue")
rm<- glmnet(Xs, Y, alpha = 0)
plot(rm, xvar = "lambda", label = TRUE)
\end{minted}

\begin{figure}[H]
    \centering
    \includegraphics[width=1\linewidth]{Capture du 2025-10-13 14-46-28.png}
\end{figure}

La ligne bleue représente le logarithme du $\lambda$ qui optimise le modèle. Ce graphique permet d'illustrer l'impact de la régularisation ($\lambda$) sur la qualité du modèle.


\begin{figure}[H]
    \centering
    \includegraphics[width=1\linewidth]{Capture du 2025-10-13 14-45-48.png}
\end{figure}

Après avoir identifié la valeur optimale du paramètre de régularisation $\lambda$, on extrait les coefficients associés du modèle. On peut ensuite déterminer quelles variables jouent le rôle le plus significatif dans la prédiction

\begin{minted}{r}
rm <- glmnet(Xs, Y, alpha = 0)
cr <- predict(rm, type = "coefficients", s = mg$lambda.min)
vr <- order(abs(cr), decreasing = TRUE)
head(vr)
\end{minted}

\texttt{[1]  1 12 10  4  7  3}

\subsection{Régression Ridge}

\end{document}
