\documentclass[12pt,a4paper]{report}

\title{Rapport TP1}
\author{Candice AUBERTIN - Ewen EXPUESTO}
\date{29 Septembre 2025}

\usepackage[a4paper, margin=1.5cm]{geometry}
\usepackage{graphicx}
\usepackage{listings}
\usepackage{amsmath}
\usepackage{color}
\usepackage{xcolor}
\usepackage{float}
\usepackage[utf8]{inputenc} % handles UTF-8 characters
\usepackage[T1]{fontenc} % better font encoding for accented characters

\setcounter{section}{0}
\renewcommand{\thesection}{\arabic{section}}

\definecolor{backgroundColour}{rgb}{0.1, 0.1, 0.1}
\definecolor{bordeaux}{rgb}{0.5, 0, 0}
\definecolor{blueVar}{rgb}{0.0, 0.0, 1.0}
\definecolor{greenKeyword}{rgb}{0.0, 0.5, 0.0}

\usepackage{minted}
\setminted{fontsize=\small, breaklines, frame=single, bgcolor=gray!5}

\begin{document}

\maketitle

\tableofcontents

\newpage
\renewcommand{\contentsname}{Table des matières}

\newpage

\section{Application: study your own data using a linear model with transformed data}

Pour cet exercice, nous avons collecté les données de la valeur en bourse d'Amazon. Nous les avons sélectionnées sur les 21 dernières années. Plusieurs catégories de données pour représenter notre étude nous étaient proposées, nous avons alors gardé la plus pertinente, le prix de clôture ajusté (Adj\_Close)

\begin{minted}{r}
tab <- read.table("~/MRR21/Files/TP1/Donnees_amazon.txt",header=TRUE, sep=";")
tab
\end{minted}

\begin{table}[h]
    \centering
    \begin{tabular}{|c|c|}
        \hline
        \textbf{Année}  & \textbf{Adj.Close} \\
        \hline
        2005 & 2.16 \\
        2006 & 2.24 \\
        2007 & 1.88 \\
        2008 & 3.88 \\
        2009 & 2.94 \\
        2010 & 6.27 \\
        2011 & 8.48 \\
        2012 & 9.72 \\
        2013 & 13.27 \\
        2014 & 17.93 \\
        2015 & 17.73 \\
        2016 & 29.35 \\
        2017 & 41.17 \\
        2018 & 72.54 \\
        2019 & 85.94 \\
        2020 & 100.44 \\
        2021 & 160.31 \\
        2022 & 149.57 \\
        2023 & 103.13 \\
        2024 & 155.2 \\
        2025 & 237.68 \\
        \end{tabular}
\end{table}

On étudie alors le modèle linéaire pour les données:

\begin{minted}{r}
modreg=lm(Adj_Close ~ Date ,tab)
summary(modreg)
\end{minted}

\begin{figure}[H]
    \centering
    \includegraphics[width=0.9\linewidth]{lm_adj_close.png}
    \caption{Ajustement du modèle linéaire sur les données d'Amazon}
    \label{fig:lm_adj_close}
\end{figure}


\begin{minted}{r}
plot(tab$Date,tab$Adj\_Close,xlab="Années",ylab="Valeurs",pch=4)
abline(modreg,col="red")
\end{minted}

\begin{figure}[H]
    \centering
    \includegraphics[width=1\linewidth]{Capture2.png}
\end{figure}

\begin{minted}{r}
plot(tab$Adj_Close,modreg$fit,xlab="Valeurs observées",ylab="Valeurs prédites",pch=4)
abline(0,1,col="red")
\end{minted}

\begin{figure}
    \centering
    \includegraphics[width=1\linewidth]{Capture du 2025-09-26 15-44-39.png}
\end{figure}

On remarque sur ces graphiques que le modèle linéaire ne convient pas, notamment en vue du coefficient $R^2$
qui vaut 0.78. L'observation des données nous amène à penser qu'un modèle exponentiel pourrait convenir. 

Nous allons donc travailler avec ln(Adj\_Close):

\begin{minted}{r}
tab$Log_Adj_Close<-log(tab$Adj_Close)
plot(tab$Date,tab$Log_Adj_Close,xlab="Années",ylab="ln(Adj_Close)",pch=4)
abline(0,1,col="red")
\end{minted}

\begin{figure}[H]
    \centering
    \includegraphics[width=1\linewidth]{Capture3.png}
\end{figure}

\begin{minted}{r}
logmodreg=lm(Log_Adj_Close~ Date,tab)
summary(logmodreg)
\end{minted}

\begin{figure}[H]
    \centering
    \includegraphics[width=1\linewidth]{lm_log_adj_close.png}
\end{figure}

\begin{minted}{r}
plot(tab$Log_Adj_Close,logmodreg$fit,xlab="Log des valeurs observées",ylab="Log des valeurs prédites",pch=4)
abline(0,1,col="red")
\end{minted}

\begin{figure}[H]
    \centering
    \includegraphics[width=1\linewidth]{Capture_log_4.png}
\end{figure}

On remarque alors que les données suivent un modèle linéaire, avec $R^2$=0.97
De plus,les résidus correspondants sont:

\begin{minted}{r}
plot(tab$Date,logmodreg$res,xlab="Années",ylab="Résidus",pch=4)
abline(h=0, col="red")
\end{minted}

\begin{figure}[H]
    \centering
    \includegraphics[width=1\linewidth]{Résidus.png}
\end{figure}

Ainsi, le modèle exponentiel correspond davantage aux données proposées. Cependant, on remarque des données légèrement moins cohérentes avec le modèle, correspondant à une expansion plus importante autour de l'année 2020. Cela peut notamment s'expliquer par l'impact de la crise sanitaire lors de cette période. Néanmoins le modèle choisit pourrait gagner en précision si on y ajoutait davantage de données sur des périodes plus courtes.


%%%%%%%%%%%%%%%%%%%%%%%%%%%%%DEUXIEME%%%%%%%%%%%%%%%%%%%%%%%%%%%%%%%%%%%%%%%%

\newpage
\section{Application: Modèle linéaire pour la production d'électricité au Mexique}

Le but est d'expliquer la production totale quotidienne d'électricité au Mexique à l'aide des variables contenues dans \texttt{Mexico\_data.csv}.

\subsection{Description des variables}
\begin{itemize}
    \item X0 : jour de l'année
    \item RH : humidité relative (\%)
    \item SSRD : rayonnement solaire incident à la surface (J.m$^{-2}$)
    \item STRD : rayonnement thermique incident à la surface (J.m$^{-2}$)
    \item T2M : température moyenne quotidienne à 2m (°C)
    \item T2Mmax : température maximale quotidienne à 2m (°C)
    \item T2Mmin : température minimale quotidienne à 2m (°C)
    \item Covid : indice de rigueur COVID-19
    \item Holidays : jours fériés, 1 = jour férié, 0 sinon
    \item DOW : jour de la semaine, 0 = lundi, 1 = mardi, ...
    \item TOY : jour de l'année (1 à 366)
    \item Total : production quotidienne d'électricité (GWh)
\end{itemize}

\subsection{Formule du modèle linéaire}

Le modèle linéaire multiple choisi est :

\begin{align*}
\mathrm{Total} =\; & \theta_0 
+ \theta_1 \mathrm{RH} 
+ \theta_2 \mathrm{SSRD} 
+ \theta_3 \mathrm{STRD} 
+ \theta_4 \mathrm{T2M} \\
& + \theta_5 \mathrm{T2Mmax} 
+ \theta_6 \mathrm{T2Mmin} 
+ \theta_7 \mathrm{Covid} 
+ \theta_8 \mathrm{Holidays} \\
& + \theta_9 \mathrm{DOW} 
+ \theta_{10} \mathrm{TOY} 
+ \varepsilon
\end{align*}


où $\theta_0$ est l'ordonnée à l'origine, $\theta_1, \dots, \theta_{10}$ sont les coefficients associés à chaque variable explicative, et $\varepsilon$ le terme d'erreur.

\subsection{Code R utilisé}

\begin{minted}{r}
# Charger les données
data <- read.csv("Mexico_data.csv")

# Identifier la variable cible et la première colonne (X0)
target_col <- "Total"
first_col <- names(data)[1]

# Exclure la cible et la première colonne des prédicteurs
predictors <- setdiff(names(data), c(target_col, first_col))

# Construire la formule du modèle
form <- as.formula(paste(target_col, "~", paste(predictors, collapse = " + ")))

# Ajuster le modèle linéaire
model <- lm(form, data = data, na.action = na.omit)
summary(model)

# Extraire valeurs observées, ajustées et résidus
obs <- model$y
fit <- fitted(model)
res <- residuals(model)

# Index pour représenter les jours (pour éviter les erreurs de type)
x_index <- seq_len(nrow(data))

# Graphique Observé vs Ajusté
plot(x_index, data[[target_col]],
     xlab = "n° du jour",
     ylab = "Production totale d'électricité (GWh)",
     main = "Données observées vs modèle ajusté")
lines(x_index, fit, col = "red")

# Graphique des résidus vs valeurs ajustées
plot(fit, res,
     xlab = "Valeurs ajustées",
     ylab = "Résidus",
     main = "Résidus du modèle")
abline(h = 0, col = "red")
\end{minted}

\subsection{Résumé des données}

Après avoir chargé les données et exécuté \texttt{summary(data)}, on obtient :

\begin{verbatim}
X0                  RH             SSRD              STRD        
 Length:1461        Min.   :28.37   Min.   : 407045   Min.   :1019376  
 Class :character   1st Qu.:50.12   1st Qu.: 719323   1st Qu.:1156350  
 Mode  :character   Median :57.66   Median : 870071   Median :1235306  
                    Mean   :57.55   Mean   : 869055   Mean   :1244835  
                    3rd Qu.:64.55   3rd Qu.:1022083   3rd Qu.:1353442  
                    Max.   :80.54   Max.   :1217895   Max.   :1431876  
      T2M            T2Mmax          T2Mmin           Covid      
 Min.   :11.23   Min.   :17.82   Min.   : 5.536   Min.   : 0.00  
 1st Qu.:18.09   1st Qu.:25.27   1st Qu.:12.064   1st Qu.: 0.00  
 Median :22.32   Median :28.72   Median :16.220   Median :33.33  
 Mean   :21.27   Mean   :27.93   Mean   :15.623   Mean   :33.00  
 3rd Qu.:24.63   3rd Qu.:30.83   3rd Qu.:19.747   3rd Qu.:63.89  
 Max.   :27.39   Max.   :33.83   Max.   :21.841   Max.   :82.41  
    Holidays            DOW         TOY            Total       
 Min.   :0.00000   Min.   :0   Min.   :  1.0   Min.   : 578.5  
 1st Qu.:0.00000   1st Qu.:1   1st Qu.: 92.0   1st Qu.: 816.0  
 Median :0.00000   Median :3   Median :183.0   Median : 871.4  
 Mean   :0.03012   Mean   :3   Mean   :183.1   Mean   : 880.7  
 3rd Qu.:0.00000   3rd Qu.:5   3rd Qu.:274.0   3rd Qu.: 958.1  
 Max.   :1.00000   Max.   :6   Max.   :366.0   Max.   :1107.4
\end{verbatim}

\subsection{Résumé du modèle linéaire}

Après avoir ajusté le modèle linéaire multiple (\texttt{lm(form, data)}), on obtient :

\begin{verbatim}
Call:
lm(formula = form, data = data, na.action = na.omit)

Residuals:
    Min      1Q  Median      3Q     Max 
-170.43  -33.41    4.57   35.44  191.36 

Coefficients:
              Estimate Std. Error t value Pr(>|t|)    
(Intercept)  3.289e+02  9.101e+01   3.614 0.000312 ***
RH          -7.449e-01  4.701e-01  -1.585 0.113241    
SSRD         1.738e-04  2.251e-05   7.722 2.12e-14 ***
STRD         4.174e-04  9.395e-05   4.442 9.58e-06 ***
T2M          2.277e-01  8.496e+00   0.027 0.978622    
T2Mmax      -4.920e+00  4.348e+00  -1.131 0.258037    
T2Mmin       6.313e+00  5.842e+00   1.081 0.280094    
Covid       -2.867e-01  4.964e-02  -5.776 9.35e-09 ***
Holidays    -8.732e+01  8.057e+00 -10.838  < 2e-16 ***
DOW         -1.305e+01  6.835e-01 -19.090  < 2e-16 ***
TOY          5.072e-02  1.716e-02   2.955 0.003172 ** 
---
Signif. codes:  0 '***' 0.001 '**' 0.01 '*' 0.05 '.' 0.1 ' ' 1

Residual standard error: 51.95 on 1450 degrees of freedom
Multiple R-squared:  0.7035, Adjusted R-squared:  0.7014 
F-statistic:   344 on 10 and 1450 DF,  p-value: < 2.2e-16
\end{verbatim}

\subsection{Visualisation des résultats}

\begin{figure}[H]
    \centering
    \includegraphics[width=1\linewidth]{results1.png}
    \caption{Production d'électricité observée et ajustée par le modèle}
\end{figure}

\begin{figure}[H]
    \centering
    \includegraphics[width=1\linewidth]{results2.png}
    \caption{Résidus du modèle linéaire en fonction des valeurs ajustées}
\end{figure}

\subsection{Commentaires et interprétation des résultats}

\begin{itemize}
    \item Le modèle linéaire multiple permet d'expliquer la production quotidienne d'électricité en fonction de plusieurs variables climatiques et sociales.
    \item D'après les valeurs de t et les p-values, les variables les plus significatives pour expliquer la production d'électricité sont \textbf{DOW} (jour de la semaine), \textbf{Holidays} (jours fériés), \textbf{SSRD} (rayonnement solaire) et \textbf{Covid} (indice de rigueur COVID-19). Ces variables ont des coefficients significatifs avec $p < 0.001$ et des valeurs absolues de t élevées.
    \item Les coefficients positifs ou négatifs indiquent le sens de l'effet sur la production totale : par exemple, \textbf{Holidays} et \textbf{DOW} ont un effet négatif important, indiquant que la production diminue pendant les jours fériés et certains jours de la semaine, tandis que \textbf{SSRD} et \textbf{STRD} ont un effet positif, montrant que la production augmente avec le rayonnement solaire et thermique.
    \item Les variables \textbf{T2M}, \textbf{T2Mmax} et \textbf{T2Mmin} ne semblent pas significatives dans ce modèle ($p > 0.1$), suggérant que la température quotidienne a moins d'impact direct sur la production totale dans cette période.
    \item Le modèle explique environ 70\% de la variance totale de la production ($R^2 = 0.7035$), ce qui indique un ajustement raisonnable mais laisse encore de la variance inexpliquée, possiblement liée à d'autres facteurs non inclus dans le dataset.
\end{itemize}



\end{document}